\documentclass[letterpaper,12pt]{article}
\usepackage{geometry}
\geometry{left=2 cm,right=2cm, top=3cm, bottom=2cm, top=2cm}
\usepackage{currvita} 
\usepackage[utf8]{inputenc}% para sacar las tildes y Ñ
\usepackage[spanish]{babel}% por si aparece algo en ingles
\usepackage{graphicx}%ingresar imagen
\usepackage{amssymb} %para utilizar signos
\title{Pseudocódigo }
\usepackage{lastpage} %ultima pagina
\usepackage{enumitem}
\date{} %para que no aparesca la fecha
\setlength{\parskip}{0.01cm}% espacio ente texto y texto
\begin{document}\ \ % lo dos \\ se utilizan hay para crear una hoja en blanco

\begin{titlepage}
	\centering
	{\scshape\LARGE\bfseries   UNIVERSIDAD DEL VALLE \par}\vspace{1cm}
	\vspace{2.5cm}
	{\scshape\Large  ACTIVIDAD 5\par}
	\vspace{2.5cm}
	{\huge Algoritmo en Pseint\\Diagramas de flujo\par} %\bfseries NEGRILLA %bf
	\vspace{2cm}
	{\Large YENIFER ANDREA BELTRAN BETIN\par} %\itshape CURSIVA
{\Large  CÓDIGO 201960124\par}% \scshape SE VEA GRANDE
	\vspace{2.5cm}
	\vfill
	ZARZAL-VALLE\par
	

	\vfill

% Bottom of the page
	{\large \today\par}
\end{titlepage}
\maketitle
 Pseudocódigo o falso Lenguaje, Es comúnmente utilizado por los programadores para omitir secciones de Código o para dar una explicación del paradigma que tomó el mismo programador para hacer sus códigos, esto quiere decir que el pseudocódigo no es programable sino facilita la programación.\\
El principal objetivo del pseudocódigo es el de representar la solución a un algoritmo de la forma más detallada posible, y a su vez lo más parecida posible al lenguaje que posteriormente se utilizará para la codificación del mismo.\\
En este informe se realiza uno de los ejemplos de una función recursiva que ofrece Pseint, el cual es calcular la  potencia, no obstante se recurrio a utilizar el siguiente Pseudocódigo:

\begin{verbatim}
// Implementación del cálculo de una potencia mediante una función recursiva
// El paso recursivo se basa en que A^B = B*(A^(B-1))
// El paso base se base en que A^0 = 1

SubProceso  resultado <- Potencia (base, exponente)
    Si exponente=0 Entonces
        resultado <- 1;
    sino 
        resultado <- base*Potencia(base,exponente-1); 
    FinSi
FinSubProceso

Proceso DosALaDiezRecursivo
    Escribir "Ingrese Base"
    Leer base
    Escribir "Ingrese Exponente"
    Leer exponente
    Escribir "El resultado es ",Potencia(base,exponente)
FinProceso
\end{verbatim}
con la informacion anterior se puede observar como el programa toma el código y realiza la funcion; este procedimiento se puede apreciar en la figura 1. 

\begin{figure}[h!]
\centering 
\includegraphics[width=\linewidth]{1.jpg}%[width=8cm, height=5cm] opciones de imagen
\caption{Pseudocódigo}
\label{Figura: 1}
\end{figure}
\vspace{10cm} %espacio
como podemos observar el código que digitamos, contiene un algoritmio principial y en base a este realiza su funcion; de igual manera al ejecutarlo nos pide una numero base y su exponente para asi poder dar un resultado.
\\
\section{Diagrama de flujo}
Pseint tiene una opción de crear un diagrama de flujo en donde se oberva desde otro punto de vista los comandos o estructuras de nuestro codigo, como vemos en la figura 2.
\begin{figure}[h!]
\centering 
\includegraphics[width=\linewidth]{2.jpg}
\caption{Diagrama de flujo}
\label{Figura: 2}
\end{figure}

con la figura anterior se puede observar como pseint, da varias opciones de diagramas de flujo segun comando o estructura; con este paso damos por finalizado esta practica.
\\ En conclusión, Pseint es un programa que brinda ayudas a escribir algoritmos utilizando un pseudo-lenguaje simple, intuitivo y en español.\\
\section{Referencias Bibliograficas}
Toda esta información es basada con los apuntes vistos en las clases virtuales de la asignatura algoritmia y programación, además se recurrio a utilizar el programa Pseint.


\end{document}